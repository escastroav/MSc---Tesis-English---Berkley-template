\chapter{Estudio de los modos de coordinación}\label{ap1}

Para el estudio de la coordinación de los ligandos que lo requerían se consideraron todos los modos posibles que guardan sentido químico y se modelaron con el método DFT expuesto en la metodología de trabajo. Como fue mencionado en el preámbulo del capítulo \ref{cap_resultados}, hay algunos resultados para los ligandos que sufrieron dos complicaciones derivadas del que disminuyeron el número de candidatos a evaluar debido a la imposibilidad de obtener valores de energía libre confiables para algunos de sus modos de coordinación. Por lo anterior, se decidió emplear un código de identificación en este apéndice para marcar si una estructura dada involucra alguno de estos dos problemas. Así, en las figuras de las secciones a continuación se indica que se obtuvo como resultado una \textbf{reducción parcial del cobre} mediante un \textbf{círculo de color amarillo} sobre la carga del complejo en cuestión, mientras que un \textbf{cambio de energía anómalo en la corrección SP} se indica con una \textbf{carga subrayada y en negrilla}. 

\section{Ligandos con átomos donadores de N, O y S}

A continuación se presentan las estructuras 2D de los modos resultantes junto con las interacciones no covalentes más importantes para cada caso, indicando la estructura que resulta ser más estable termodinámicamente en color verde y omitiendo del análisis a las moléculas que involucran resultados inconsistentes marcadas con el código de identificación.

\subsection{Derivados tetraazamacrocíclicos}

Las Figs. \ref{estcoord_1_3} y \ref{estcoord_4_7} muestran que la desprotonación de los anillos en los derivados tetraazamacrocíclicos da lugar a deltas de energía libre positivos en todos los casos para los complejos formados con ambos cationes $Cu^{1+}$ y $Cu^{2+}$, por lo que las estructuras más estables resultan ser aquellas con anillos completamente protonados. Se observa también que añadir un metilpicolinato a la estructura \textbf{1} original hace que los modos de coordinación más estables para ambos cationes de cobre cambien de 4N a 5N para el ligando \textbf{2}, lo que ocurre gracias a la coordinación del anillo de piridina del fragmento el cual a su vez induce una geometría piramidal. Por otro lado, añadir un segundo metilpicolinato en el amino opuesto como sucede con la estructura \textbf{3} modifica la coordinación con el $Cu^{1+}$ a un modo 3N1O, incluyendo a un oxígeno del grupo carboxilato en la coordinación y rompiendo las interacciones del metal con los aminos sustituidos.\\

Respecto a los macrociclos de más miembros, encontramos que el ligando \textbf{4} mantiene modos de coordinación 4N para ambos cationes de cobre, los cuales cambian al añadir un fragmento de metilpicolinato hacia estructuras pentacoordinadas 5N con el anillo de piridina, como se puede observar para el caso del ligando \textbf{6}. Adicionalmente, las estructuras estables de \textbf{7} muestran que un segundo metilpicolinato en el amino opuesto produce estructuras planares 4N que no exhiben coordinación de parte de ninguno de los dos anillos de piridina, ya que aunque ambos N se encuentran orientados hacia el centro metálico sus distancias son mayores a 3,0 \AA. Por otro lado, el ligando \textbf{5} muestra que la metilación de dos aminos opuestos y la adición de un fragmento de N-(2-piridin-2-iletil)acetamida produce una coordinación estable del carbonilo de la estructura generando coordinaciones 3N1O con el $Cu^{1+}$ y 4N1O con el $Cu^{2+}$.

\hspace{0pt}
\vfill
\begin{figure}[ht] 
\centering
\includegraphics[scale=0.20]{images/Estabilidad_estilb/1_3.png}
\caption{Modos de coordinación y $\Delta G^{\circ}$ (kcal mol$^{-1}$) para los complejos: \textbf{a)} \textbf{1}$\ \Rightarrow Cu^{1+/2+}$, \textbf{b)} \textbf{2}$\ \Rightarrow Cu^{1+/2+}$, \textbf{c)} \textbf{3}$\ \Rightarrow Cu^{1+}$, y \textbf{d)} \textbf{3}$\ \Rightarrow Cu^{2+}$.}
\label{estcoord_1_3}
\end{figure}
\vfill
\hspace{0pt}

\clearpage

\hspace{0pt}
\vfill
\begin{figure}[ht] 
\centering
\centerline{\includegraphics[scale=0.20]{images/Estabilidad_estilb/4_7.png}}
\caption{Modos de coordinación y $\Delta G^{\circ}$ (kcal mol$^{-1}$) para los complejos: \textbf{a)} \textbf{4}$\ \Rightarrow Cu^{1+/2+}$, \textbf{b)} \textbf{5}$\ \Rightarrow Cu^{1+}$, \textbf{c)} \textbf{5}$\ \Rightarrow Cu^{2+}$, \textbf{d)} \textbf{6}$\ \Rightarrow Cu^{1+/2+}$, y \textbf{e)} \textbf{7}$\ \Rightarrow Cu^{1+/2+}$.}\label{estcoord_4_7}
\end{figure}
\vfill
\hspace{0pt}

\clearpage

\subsection{Derivados de aminopiridina}

La Fig. \ref{estcoord_8_9} muestra que los modos de coordinación con relación 1:2 más estables para el ligando \textbf{8} son 4N para ambos cationes de cobre y que estos no exhiben desprotonación alguna. Se evidencia además que sus complejos con relación 1:1 más estables se obtienen completando las vacancias de coordinación con una molécula de agua para el caso del $Cu^{1+}$, y con dos moléculas de agua para el $Cu^{2+}$. Respecto al ligando \textbf{9}, se encuentra que los modos más estables de este exhiben la misma coordinación tetradentada 4N sin desprotonación para ambos cationes de cobre.

\hspace{0cm}
\vfill
\begin{figure}[ht] 
\centering
\includegraphics[scale=0.20]{images/Estabilidad_estilb/8_9.png}
\caption{Modos de coordinación y $\Delta G^{\circ}$ (kcal mol$^{-1}$) para los complejos: \textbf{a)} \textbf{8}$\ \Rightarrow Cu^{1+}$, \textbf{b)} \textbf{8}$\ \Rightarrow Cu^{2+}$, y \textbf{c)} \textbf{9}$\ \Rightarrow Cu^{1+/2+}$.}\label{estcoord_8_9}
\end{figure}
\vfill
\hspace{0cm}

\clearpage

\subsection{Análogos de estilbeno}

Un único un modo de coordinación fue considerado para los complejos con relación 1:2 formados con los ligandos \textbf{13}-\textbf{17}, y \textbf{19}, el cual consiste en la desprotonación del nitrógeno $sp^{3}$ coordinante al tratarse de un átomo dador muy afín al cobre. En contraste, varios modos de coordinación diferentes se modelaron para los ligandos \textbf{18} y \textbf{20} debido a la presencia de un grupo hidroxilo potencialmente desprotonable en sus estructuras. Se encuentra que esta familia de ligandos cumple en todos los casos que sus complejos 1:1 más estables llenan su vacancia con una molécula de agua para el caso del $Cu^{1+}$, y dos para el $Cu^{2+}$. Así, a continuación se discuten las implicaciones de los modos de coordinación con relaciones 1:2.\\

Como se puede visualizar en la Fig. \ref{estcoord_12_16}, los ligandos \textbf{12}, \textbf{13}, \textbf{15} y \textbf{16} presentan modos 4N con coordinación del anillo de piridina para ambos cationes de cobre, mientras que la estructura no sustituida \textbf{14} exhibe una coordinación 2N con geometría lineal para la especie $Cu^{1+}$, y 4N para el $Cu^{2+}$, mostrando que sustituir el anillo de piridina con distintos grupos permite modular su carácter dador y modificar los modos de coordinación obtenidos.\\

La Fig. \ref{estcoord_17y19} muestra con el caso del ligando \textbf{17} que extender el anillo aromático de piridina de la estructura \textbf{13} en uno del tipo quinolina no favorece un modo de coordinación 4N para el complejo de la especie $Cu^{1+}$, produciéndose un modo 2N con geometría lineal. En contraste, con las sustituciones de grupos metoxi en el ligando \textbf{19} se obtienen estructuras estables 4N para ambos cationes del cobre, efecto que puede deberse a la naturaleza electrodadora del metoxi la cual a su vez aumenta el carácter dador del nitrógeno presente en la quinolina. En el tratamiento del ligando \textbf{18} en la Fig. \ref{estcoord_18} se evidencia que cambiar los grupos metoxi por grupos donadores hidroxilo resulta en un modo de coordinación estable 2N2O con la estructura de tipo 8-hidroxiquinolina para la especie $Cu^{1+}$. El ligando \textbf{20} presentado en la Fig. \ref{estcoord_20} es un homólogo de la estructura \textbf{18}, cuya diferencia radica en el grupo imino, para el cual se encuentra un modo de coordinación 4N con el catión $Cu^{1+}$ en el que solo un ligando exhibe desprotonación en su hidroxilo. Adicionalmente, la estructura resultante se ve estabilizada por la formación de un enlace de hidrógeno intramolecular entre los fragmentos de 8-hidroxiquinolina de los dos ligandos. En contraste, el ligando \textbf{21} mostrado en la Fig. \ref{estcoord_21y22} no ofrece cambio alguno en la coordinación 4N respecto a su homólogo \textbf{19}, y el ligando tetradentado \textbf{22} resulta en un modo 2N2O estable para los dos cationes de cobre en el cual se exhibe únicamente la desprotonación del grupo fenólico.

\clearpage

\begin{figure}[ht] 
\centering
\centerline{\includegraphics[scale=0.20]{images/Estabilidad_estilb/12_16.png}}
\caption{Modos de coordinación y $\Delta G^{\circ}$ (kcal mol$^{-1}$) para los complejos: \textbf{a)} \textbf{12}$\ \Rightarrow Cu^{1+/2+}$, \textbf{b)} \textbf{13}$\ \Rightarrow Cu^{1+/2+}$, \textbf{c)} \textbf{14}$\ \Rightarrow Cu^{1+}$, \textbf{d)} \textbf{14}$\ \Rightarrow Cu^{2+}$, \textbf{e)} \textbf{15}$\ \Rightarrow Cu^{1+/2+}$, y \textbf{f)} \textbf{16}$\ \Rightarrow Cu^{1+/2+}$.}
\label{estcoord_12_16}
\end{figure}

\begin{figure}[ht] 
\centering
\includegraphics[scale=0.20]{images/Estabilidad_estilb/17y19.png}
\caption{Modos de coordinación y $\Delta G^{\circ}$ (kcal mol$^{-1}$) para los complejos: \textbf{a)} \textbf{17}$\ \Rightarrow Cu^{1+/2+}$, y \textbf{b)} \textbf{19}$\ \Rightarrow Cu^{1+/2+}$.}
\label{estcoord_17y19}
\end{figure}

\begin{figure}[ht] 
\centering
\includegraphics[scale=0.20]{images/Estabilidad_estilb/18.png}
\caption{Modos de coordinación y $\Delta G^{\circ}$ (kcal mol$^{-1}$) para los complejos: \textbf{a)} \textbf{18}$\ \Rightarrow Cu^{1+}$, y \textbf{b)} \textbf{18}$\ \Rightarrow Cu^{2+}$.}
\label{estcoord_18}
\end{figure}

\begin{figure}[ht] 
\centering
\centerline{\includegraphics[scale=0.20]{images/Estabilidad_estilb/20.png}}
\caption{Modos de coordinación y $\Delta G^{\circ}$ (kcal mol$^{-1}$) para los complejos: \textbf{a)} \textbf{20}$\ \Rightarrow Cu^{1+}$, y \textbf{b)} \textbf{20}$\ \Rightarrow Cu^{2+}$.}
\label{estcoord_20}
\end{figure}

\begin{figure}[ht] 
\centering
\centerline{\includegraphics[scale=0.20]{images/Estabilidad_estilb/21y22.png}}
\caption{Modos de coordinación y $\Delta G^{\circ}$ (kcal mol$^{-1}$) para los complejos: \textbf{a)} \textbf{21}$\ \Rightarrow Cu^{1+/2+}$, y \textbf{b)} \textbf{22}$\ \Rightarrow Cu^{1+/2+}$.}
\label{estcoord_21y22}
\end{figure}

\clearpage

\subsection{Derivados de IMPY}

Los resultados obtenidos para esta familia de ligandos sugieren que los modos de coordinación más estables con relación 1:1 se obtienen al completar las vacantes de coordinación con una molécula de agua para la especie de $Cu^{1+}$. La Fig. \ref{estcoord_23_25} muestra que para el ligando \textbf{23} se encuentra un modo de coordinación estable 2N de geometría lineal para la especie $Cu^{1+}$, sin desprotonación de los grupos hidroxilo del fragmento heterocíclico, mientras que para el caso del $Cu^{2+}$ se encontró una coordinación 2N2O que involucra la desprotonación de ambos fragmentos. Para el ligando \textbf{24} se evidencia un modo de coordinación estable 4N para ambos cationes de cobre, y, para el ligando \textbf{25}, se puede ver en la figura que todos sus resultados están comprometidos con la problemática de la corrección energética mediante SP, razón por la cual se descarta directamente de los análisis posteriores de SRP y afinidad.

\hspace{0cm}
\vfill
\begin{figure}[ht] 
\centering
\centerline{\includegraphics[scale=0.20]{images/Estabilidad_estilb/23_25.png}}
\caption{Modos de coordinación y $\Delta G^{\circ}$ (kcal mol$^{-1}$) para los complejos: \textbf{a)} \textbf{23}$\ \Rightarrow Cu^{1+}$, \textbf{b)} \textbf{23}$\ \Rightarrow Cu^{2+}$, \textbf{c)} \textbf{24}$\ \Rightarrow Cu^{1+/2+}$, y \textbf{d)} \textbf{25}$\ \Rightarrow Cu^{1+/2+}$.}
\label{estcoord_23_25}
\end{figure}
\vfill
\hspace{0cm}

\subsection{Análogos de PiB}

Para esta familia de ligandos se evidencia nuevamente que los modos de coordinación en relación 1:1 más estables para las especies de $Cu^{1+}$ y $Cu^{2+}$ se consiguen al completar las vacantes de coordinación con una y dos moléculas de agua, respectivamente. La Fig. \ref{estcoord_26} muestra que para el ligando \textbf{26} el modo de coordinación con relación 1:2 más estable con el catión $Cu^{1+}$ es de tipo 2N2O sin desprotonación alguna, mientras que para el $Cu^{2+}$ se consigue una mayor estabilidad con un modo 2N2O con desprotonaciones en los grupos fenólicos.\\

El efecto de sustituir el anillo fenólico de \textbf{26} en posición \textit{para} con grupos electroatractores se puede evaluar con las Figs. \ref{estcoord_27} y \ref{estcoord_28}. Aquí, el ligando \textbf{27} muestra que añadir un cloro a la estructura cambia la coordinación del catión $Cu^{1+}$ a un modo tricoordinado 2N1O que involucra la desprotonación de un grupo fenólico, mientras que el ligando \textbf{28} muestra que aumentar el carácter electroatractor del sustituyente (con un grupo nitro en este caso) induce para el $Cu^{1+}$ un modo de coordinación 2N2O en el que ambos grupos fenólicos se encuentran desprotonados. Cabe mencionar, además, que en ninguno de los casos anteriores se establece la coordinación de los heteroátomos del fragmento de 1,3-benzotiazol presente en la estructura, sino únicamente del nitrógeno del grupo imino. Finalmente, las Figs. \ref{estcoord_29y30} y \ref{estcoord_31} muestran los casos de descarte directo de las estructuras \textbf{29}, \textbf{30} y \textbf{31} de los análisis de SRP y afinidad, dado que sus resultados involucran la problemática de la corrección energética con SP mencionada previamente.

\clearpage

\hspace{0cm}
\vfill
\begin{figure}[ht] 
\centering
\includegraphics[scale=0.20]{images/Estabilidad_estilb/26.png}
\caption{Modos de coordinación y $\Delta G^{\circ}$ (kcal mol$^{-1}$) para los complejos: \textbf{a)} \textbf{26}$\ \Rightarrow Cu^{1+}$, y \textbf{b)} \textbf{26}$\ \Rightarrow Cu^{2+}$.}
\label{estcoord_26}
\end{figure}
\vfill
\hspace{0cm}

\begin{figure}[ht] 
\centering
\includegraphics[scale=0.20]{images/Estabilidad_estilb/27.png}
\caption{Modos de coordinación y $\Delta G^{\circ}$ (kcal mol$^{-1}$) para los complejos: \textbf{a)} \textbf{27}$\ \Rightarrow Cu^{1+}$, y \textbf{b)} \textbf{27}$\ \Rightarrow Cu^{2+}$.}
\label{estcoord_27}
\end{figure}

\begin{figure}[ht] 
\centering
\centerline{\includegraphics[scale=0.20]{images/Estabilidad_estilb/28.png}}
\caption{Modos de coordinación y $\Delta G^{\circ}$ (kcal mol$^{-1}$) para los complejos: \textbf{a)} \textbf{28}$\ \Rightarrow Cu^{1+}$, y \textbf{b)} \textbf{28}$\ \Rightarrow Cu^{2+}$.}
\label{estcoord_28}
\end{figure}

\begin{figure}[ht] 
\centering
\includegraphics[scale=0.20]{images/Estabilidad_estilb/29y30.png}
\caption{Modos de coordinación y $\Delta G^{\circ}$ (kcal mol$^{-1}$) para los complejos: \textbf{a)} \textbf{29}$\ \Rightarrow Cu^{1+/2+}$, y \textbf{b)} \textbf{30}$\ \Rightarrow Cu^{1+/2+}$.}
\label{estcoord_29y30}
\end{figure}

\begin{figure}[ht] 
\centering
\centerline{\includegraphics[scale=0.20]{images/Estabilidad_estilb/31.png}}
\caption{Modos de coordinación y $\Delta G^{\circ}$ (kcal mol$^{-1}$) para los complejos: \textbf{a)} \textbf{31}$\ \Rightarrow Cu^{1+}$, y \textbf{b)} \textbf{31}$\ \Rightarrow Cu^{2+}$.}
\label{estcoord_31}
\end{figure}

\clearpage

Aunque los resultados mostrados previamente incluyen deltas de descoordinación de ligandos para evaluar qué relación metal-ligando presenta una mayor estabilidad, es preciso tener en cuenta que todos los valores están calculados para condiciones estándar de reacción. Así, en condiciones experimentales y biológicas se pueden obtener complejos con ambas relaciones dependiendo de factores como el método de preparación del complejo o las condiciones de concentración del compuesto a nivel biológico. Por esta razón, en el presente trabajo se consideraron ambas relaciones 1:2 y 1:1 para la evaluación de las propiedades farmacológicas de este grupo bajo estudio.\\

Finalmente, omitiendo las estructuras cuyos resultados termodinámicos se ven obstaculizados por problemas de reducción parcial del centro metálico y de corrección energética vía SP, se redujo el conjunto original de ligandos a un subconjunto de 22 complejos de coordinación con relación 1:2 y 6 complejos con relación 1:1 para los cálculos subsecuentes de SRP y afinidad. Es necesario mencionar también que como resultado de este análisis fue posible identificar que la problemática asociada a la corrección con SP es predominante en moléculas con sustituyentes de yodo, siendo el caso de las estructuras \textbf{25}, \textbf{29}, \textbf{30} y \textbf{31}.

\section{Ligandos tipo pinza}

El primer grupo de moléculas consideradas en este estudio de los modos de coordinación fueron aquellos ligandos con grupos laterales OH y NHPh potencialmente desprotonables durante el proceso de coordinación. Sus desprotonaciones posibles son mostradas en la Fig. \ref{desprotonaciones_2DE}, mientras que los deltas de energía libre se agrupan en la Tabla \ref{deltas_desp_2DE}.\\ 
\begin{figure}[ht] 
\centering
\includegraphics[scale=0.20]{images/Fig_resultados/desprotonaciones_2DE.png}
\caption{Modos de coordinación posibles para los ligandos \textbf{34}-\textbf{37} y \textbf{42}-\textbf{45}.}\label{desprotonaciones_2DE}
\end{figure}

Los identificadores en la tabla muestran que la mayor parte de ligandos se ven comprometidos por problemas de reducción parcial del centro metálico o de corrección de energía via SP, por lo que solo los ligandos \textbf{43} y \textbf{44} de este grupo pudieron ser considerados en el análisis presentado en el capítulo de resultados. Podemos observar en los resultados de los delta de desprotonación obtenidos para estos dos ligandos que su especie más estable es aquella que exhibe únicamente la desprotonación central.

\clearpage

\begin{table}[t]
\begin{center}
\caption{Deltas de energía libre para las desprotonaciones de coordinación posibles de los ligandos tipo pinza 34-37 y 42-45.}
\begin{threeparttable}
\begin{tabular}{c|cc|cc}
\hline
\multirow{2}{*}{\textbf{Ligando}} & \multicolumn{2}{c|}{\textbf{$\boldsymbol{Cu^{1+}}$}}           & \multicolumn{2}{c}{\textbf{$\boldsymbol{Cu^{2+}}$}}            \\
                                  & \textbf{Lateral} & \textbf{Doble} & \textbf{Lateral} & \textbf{Doble} \\ \hline
34                                & 15,2                  & 19,6                 & -1,4$^{a}$                 & 3,4                  \\
35                                & 15,9                  & 20,0                 & -19,3$^{b}$                & 4,1                  \\
36                                & 12,2                  & 17,4                 & -1,8$^{a}$                 & 0,5                  \\
37                                & 6,6$^{b}$                  & 20,0$^{b}$                & 33,8$^{b}$                 & -33,1$^{b}$               \\
42                                & 27,6                  & 30,7                 & -13,4$^{b}$                & 19,4                 \\
43                                & 27,4                  & 30,5                 & 13,7                  & 19,8                 \\
44                                & 26,2                  & 27,1                 & 10,8                  & 15,9                 \\
45                                & 17,3$^{b}$                 & 27,9$^{b}$                & 12,4$^{b}$                 & 29,0$^{a,b}$               \\ \hline
\end{tabular}
\begin{tablenotes}
    \item[a] \footnotesize{Valores obtenidos a partir de resultados con reducción parcial del centro de cobre}.
    \item[b] \footnotesize{Valores obtenidos a partir de resultados con valores anómalos en la corrección con SP}.
\end{tablenotes}
\end{threeparttable}
\label{deltas_desp_2DE}
\end{center}
\end{table}

Por otro lado, la optimización de algunos ligandos tipo pinza resultaron en modos de coordinación con interacciones anómalas con su agua de coordinación al considerar la especie de $Cu^{1+}$. Para el caso de los ligandos \textbf{32} y \textbf{33} se encontró que estos exhiben geometrías piramidales poco favorecidas para la coordinación de la molécula de agua en el centro metálico. De forma similar, se encontró que para los ligandos \textbf{47}-\textbf{49} un alto impedimento estérico en las geometrías impide la correcta coordinación de la molécula de agua resultando para la mayoría de casos en la descoordinación de la misma. Finalmente, un efecto opuesto se evidenció en los ligandos \textbf{57}, \textbf{59} y \textbf{64}, los cuales presentaron modos con descoordinación del átomo dador central del ligando debido a la alta tensión angular en sus estructuras.\\

Por lo anterior se realizó un análisis de la coordinación de estos ligandos evaluando el efecto de las moléculas de agua. Sus resultados se muestran en la Fig. \ref{coordinaciones_pincer} y en ella se ilustran los modos de coordinación más estables resaltados en color verde. Para este conjunto de ligandos mencionado se evidencia que se consigue llegar a estructuras más estables al retirar las aguas del sistema en todos los casos que estas exhiben geometrías anómalas o se encuentran directamente descoordinadas, siendo la única excepción a esto la especie de $Cu^{2+}$ del ligando \textbf{33}, la cual se mantiene más estable al conservar su molécula de agua coordinada. Finalmente, se encuentra que los intentos de completar la tetracordinación con una molécula de agua adicional para los casos en que ocurre la descoordinación del átomo dador central resulta en deltas de energía libre positivos o en estructuras sin beneficio alguno, tal como es el caso del ligando \textbf{57}, para el cual la molécula de agua extra no entra en la coordinación del metal a pesar de exhibir un delta negativo.\\

\hspace{0cm}
\vfill
\begin{figure}[ht] 
\centering
\centerline{\includegraphics[scale=0.20]{images/Fig_resultados/coordinaciones_Pincer.png}}
\caption{Evaluación de los modos de coordinación con el catión $Cu^{1+}$ para los ligandos: \textbf{a)} \textbf{32}, \textbf{b)} \textbf{33}, \textbf{c)} \textbf{46}, \textbf{d)} \textbf{47}, \textbf{e)} \textbf{48}, \textbf{f)} \textbf{49}, \textbf{g)} \textbf{57}, \textbf{h)} \textbf{59}, y \textbf{i)} \textbf{64}. Valores de $\Delta G^{\circ}$ dados en kcal mol$^{-1}$.}
\label{coordinaciones_pincer}
\end{figure}
\hspace{0cm}
\vfill